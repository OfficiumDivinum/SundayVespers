\no{2}\evenverse Quia ipse super mária fun\textbf{dá}vit \textbf{e}um:~\*\\
\evenverse et super flúmina præpa\textbf{rá}vit \textbf{e}um.\\
\no{3}\oddverse Quis ascéndet in \textbf{mon}tem \textbf{Dó}\textbf{mi}ni?~\*\\
\oddverse aut quis stabit in loco \textbf{san}cto \textbf{e}jus?\\
\no{4}\evenverse Innocens mánibus et mundo corde,~+\\
\evenverse  qui non accépit in vano \textbf{á}nimam \textbf{su}am,~\*\\
\evenverse nec jurávit in dolo \textbf{pró}ximo \textbf{su}o.\\
\no{5}\oddverse Hic accípiet benedicti\textbf{ó}nem a \textbf{Dó}\textbf{mi}no:~\*\\
\oddverse et misericórdiam a Deo, salu\textbf{tá}ri \textbf{su}o.\\
\no{6}\evenverse Hæc est generátio quæ\textbf{rén}tium \textbf{e}um,~\*\\
\evenverse quæréntium fáciem \textbf{De}i \textbf{Ja}cob.\\
\no{7}\oddverse Attóllite portas, príncipes, vestras,~+\\
\oddverse  et elevámini, portæ \textbf{æ}ter\textbf{ná}les:~\*\\
\oddverse et intro\textbf{í}bit Rex \textbf{gló}riæ.\\
\no{8}\evenverse Quis est iste Rex glóriæ?~+\\
\evenverse  Dóminus \textbf{for}tis et \textbf{po}tens:~\*\\
\evenverse Dóminus \textbf{po}tens in \textbf{prǽ}lio.\\
\no{9}\oddverse Attóllite portas, príncipes, vestras,~+\\
\oddverse  et elevámini, portæ \textbf{æ}ter\textbf{ná}les:~\*\\
\oddverse et intro\textbf{í}bit Rex \textbf{gló}riæ.\\
\no{10}\evenverse Quis est \textbf{i}ste Rex \textbf{gló}\textbf{ri}æ?~\*\\
\evenverse Dóminus virtútum ipse \textbf{est} Rex \textbf{gló}riæ.\\
\no{11}\oddverse Glória \textbf{Pa}tri, et \textbf{Fí}\textbf{li}o,~\*\\
\oddverse et Spi\textbf{rí}tui \textbf{San}cto.\\
\no{12}\evenverse Sicut erat in princípio, et \textbf{nunc}, et \textbf{sem}per,~\*\\
\evenverse et in sǽcula sæcu\textbf{ló}rum. \textbf{A}men.\\
