\no{2}\evenverse In salícibus in médio \textbf{e}jus,~\*\\
\evenverse suspéndimus \textbf{ór}gana \textbf{no}stra.\\
\no{3}\oddverse Quia illic interrogavérunt nos, qui captívos du\textbf{xé}runt nos,~\*\\
\oddverse verba \textbf{can}ti\textbf{ó}num.\\
\no{4}\evenverse Et qui abdu\textbf{xé}runt nos:~\*\\
\evenverse Hymnum cantáte nobis de \textbf{cán}ticis \textbf{Si}on.\\
\no{5}\oddverse Quómodo cantábimus cánticum \textbf{Dó}mini~\*\\
\oddverse in terra \textbf{a}li\textbf{é}na?\\
\no{6}\evenverse Si oblítus fúero tui, Je\textbf{rú}salem,~\*\\
\evenverse oblivióni detur \textbf{déx}tera \textbf{me}a.\\
\no{7}\oddverse Adhǽreat lingua mea fáucibus \textbf{me}is,~\*\\
\oddverse si non me\textbf{mí}nero \textbf{tu}i.\\
\no{8}\evenverse Si non proposúero Je\textbf{rú}salem,~\*\\
\evenverse in princípio læ\textbf{tí}tiæ \textbf{me}æ.\\
\no{9}\oddverse Memor esto, Dómine, filiórum \textbf{E}dom,~\*\\
\oddverse in \textbf{di}e Je\textbf{rú}salem.\\
\no{10}\evenverse Qui dicunt: Exinaníte, exina\textbf{ní}te~\*\\
\evenverse usque ad funda\textbf{mén}tum in \textbf{e}a.\\
\no{11}\oddverse Fília Babylónis \textbf{mí}sera:~\*\\
\oddverse beátus, qui retríbuet tibi retributiónem tuam, quam retribu\textbf{í}sti \textbf{no}bis.\\
\no{12}\evenverse Beátus qui te\textbf{né}bit,~\*\\
\evenverse et allídet párvulos \textbf{tu}os ad \textbf{pe}tram.\\
\no{13}\oddverse Glória Patri, et \textbf{Fí}lio,~\*\\
\oddverse et Spi\textbf{rí}tui \textbf{San}cto.\\
\no{14}\evenverse Sicut erat in princípio, et nunc, et \textbf{sem}per,~\*\\
\evenverse et in sǽcula sæcu\textbf{ló}rum. \textbf{A}men.\\
