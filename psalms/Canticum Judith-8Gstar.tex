{2.~}Adonái, Dómine, magnus es tu, et præclárus in virtúte \textbf{tu}a,~* et quem superáre \textit{ne}\textit{mo} \textbf{po}test.\\
{3.~}Tibi sérviat omnis creatúra \textbf{tu}a:~* quia dixí\textit{sti}, \textit{et} \textbf{fa}cta sunt:\\
{4.~}Misísti spíritum tuum, et cre\textbf{á}ta sunt,~* et non est qui resístat \textit{vo}\textit{ci} \textbf{tu}æ.\\
{5.~}Montes a fundaméntis movebúntur cum \textbf{a}quis:~* petræ, sicut cera, liquéscent ante fá\textit{ci}\textit{em} \textbf{tu}am.\\
{6.~}Qui autem \textbf{ti}ment te,~* magni erunt apud \textit{te} \textit{per} \textbf{óm}nia.\\
{7.~}Væ genti insurgénti super genus meum: Dóminus enim omnípotens vindicábit in \textbf{e}is,~* in die judícii visi\textit{tá}\textit{bit} \textbf{il}los.\\
{8.~}Dabit enim ignem, et vermes in carnes e\textbf{ó}rum,~* ut urántur, et séntiant usque in \textit{sem}\textit{pi}\textbf{tér}num.\\
{9.~}Glória Patri, et \textbf{Fí}lio,~* et Spirí\textit{tu}\textit{i} \textbf{San}cto.\\
{10.~}Sicut erat in princípio, et nunc, et \textbf{sem}per,~* et in sǽcula sæcu\textit{ló}\textit{rum}. \textbf{A}men.\\
