{2.~}Quia ipse super mária fundávit \textbf{e}um:~* et super flúmina præpará\textit{vit} \textbf{e}um.\\
{3.~}Quis ascéndet in montem \textbf{Dó}mini?~* aut quis stabit in loco san\textit{cto} \textbf{e}jus?\\
{4.~}Innocens mánibus et mundo corde,~† qui non accépit in vano ánimam \textbf{su}am,~* nec jurávit in dolo próxi\textit{mo} \textbf{su}o.\\
{5.~}Hic accípiet benedictiónem a \textbf{Dó}mino:~* et misericórdiam a Deo, salutá\textit{ri} \textbf{su}o.\\
{6.~}Hæc est generátio quæréntium \textbf{e}um,~* quæréntium fáciem De\textit{i} \textbf{Ja}cob.\\
{7.~}Attóllite portas, príncipes, vestras,~† et elevámini, portæ æter\textbf{ná}les:~* et introíbit \textit{Rex} \textbf{gló}\textbf{ri}æ.\\
{8.~}Quis est iste Rex glóriæ?~† Dóminus fortis et \textbf{po}tens:~* Dóminus potens \textit{in} \textbf{prǽ}\textbf{li}o.\\
{9.~}Attóllite portas, príncipes, vestras,~† et elevámini, portæ æter\textbf{ná}les:~* et introíbit \textit{Rex} \textbf{gló}\textbf{ri}æ.\\
{10.~}Quis est iste Rex \textbf{gló}riæ?~* Dóminus virtútum ipse est \textit{Rex} \textbf{gló}\textbf{ri}æ.\\
{11.~}Glória Patri, et \textbf{Fí}lio,~* et Spirítu\textit{i} \textbf{San}cto.\\
{12.~}Sicut erat in princípio, et nunc, et \textbf{sem}per,~* et in sǽcula sæculó\textit{rum}. \textbf{A}men.\\
