{2.~}Quia ipse super mária fundávit \textbf{e}um:~* et super flúmina præpa\textbf{rá}vit \textbf{e}um.\\
{3.~}Quis ascéndet in montem \textbf{Dó}mini?~* aut quis stabit in loco \textbf{san}cto \textbf{e}jus?\\
{4.~}Innocens mánibus et mundo corde,~† qui non accépit in vano ánimam \textbf{su}am,~* nec jurávit in dolo \textbf{pró}ximo \textbf{su}o.\\
{5.~}Hic accípiet benedictiónem a \textbf{Dó}mino:~* et misericórdiam a Deo, salu\textbf{tá}ri \textbf{su}o.\\
{6.~}Hæc est generátio quæréntium \textbf{e}um,~* quæréntium fáciem \textbf{De}i \textbf{Ja}cob.\\
{7.~}Attóllite portas, príncipes, vestras,~† et elevámini, portæ æter\textbf{ná}les:~* et intro\textbf{í}bit Rex \textbf{gló}riæ.\\
{8.~}Quis est iste Rex glóriæ?~† Dóminus fortis et \textbf{po}tens:~* Dóminus \textbf{po}tens in \textbf{prǽ}lio.\\
{9.~}Attóllite portas, príncipes, vestras,~† et elevámini, portæ æter\textbf{ná}les:~* et intro\textbf{í}bit Rex \textbf{gló}riæ.\\
{10.~}Quis est iste Rex \textbf{gló}riæ?~* Dóminus virtútum ipse \textbf{est} Rex \textbf{gló}riæ.\\
{11.~}Glória Patri, et \textbf{Fí}lio,~* et Spi\textbf{rí}tui \textbf{San}cto.\\
{12.~}Sicut erat in princípio, et nunc, et \textbf{sem}per,~* et in sǽcula sæcu\textbf{ló}rum. \textbf{A}men.\\
